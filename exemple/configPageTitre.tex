% AVERTISSEMENT : toujours vous référer au document ``Explication de la page
% titre 2023'' disponible sur le site web de l'ÉTS :
% https://www.etsmtl.ca/personnel-enseignant-examens-finaux


%%% Les tailles des boites sont fixes et ne tiennent pas compte de ce
%%% qui y est écrit. Si vous souhaite modifier la taille d'une boite, 
%%% utilisez les commandes suivantes:
%\setHauteurBoiteOuEcrireLesReponses{2.0}
%\setHauteurBoiteCalculatrice{1.4}
%\setHauteurBoiteDocumentation{1.4}
%\setHauteurBoiteAnnexes{0.9}
%\setHauteurBoiteDirectivesParticulieres{2.0}

%%%  Sélectionner le type d'examen
% Note : si l'examen est en deux parties, chaque partie doit avoir sa page titre.
%\setType{EXAMEN INTRA}
\setType{EXAMEN INTRA -- Partie 1 de 2}
%\setType{EXAMEN INTRA -- Partie 2 de 2}
%\setType{EXAMEN INTRA DIFFÉRÉ}
%\setType{EXAMEN INTRA DIFFÉRÉ -- Partie 1 de 2}
%\setType{EXAMEN INTRA DIFFÉRÉ -- Partie 2 de 2}
%\setType{EXAMEN FINAL}
%\setType{EXAMEN FINAL -- Partie 1 de 2}
%\setType{EXAMEN FINAL -- Partie 2 de 2}
%\setType{EXAMEN FINAL DIFFÉRÉ}
%\setType{EXAMEN FINAL DIFFÉRÉ -- Partie 1 de 2}
%\setType{EXAMEN FINAL DIFFÉRÉ -- Partie 2 de 2}


%%%  Quand l'examen aura-t-il lieu ?
\setAnnee{2023}
\setSaison{Automne} % Choisir parmi : Hiver, Été, Automne
\setDate{8 août} % Jour et mois (l'année est déjà spécifiée avec \setAnnee)
\setHeure{9h00} % Début de l'examen
\setDuree{3h00} % Je pense que c'est évident.

% Dans de rares cas, un examen a lieu à une date dont l'année est différente de
% celle de la session. Typiquement, il s'agit des examens de reprises effectués
% en janvier.
%\setAnneeDifferenteDeLaSession{2024}


%%%  Informations générales
\setSigleEtTitre{MAT000 -- Éléments de mathématiques}
\setGroupes{01 et 10}
\setEnseignants{
  Alan Smithee
}


%%  Format de l'examen, sélectionner
%\setOuEcrireLesReponses{dans le cahier d'examen standard ÉTS.}
\setOuEcrireLesReponses{sur ce questionnaire.}
%\setOuEcrireLesReponses{sur ce questionnaire pour les questions ?? à ?? et dans le cahier d'examen standard ÉTS pour les question ?? à ??.}

%% facultatif : ajouter des précisions quant à l'endroit où écrire les réponses
%\setOuEcrireLesReponsesPrecisions{Pour les graphiques, utilisez des crayons de couleur.}


% Nombre de question et nombre de pages. Si l'examen est sur ENAQuiz, il faut
\setNbQuestions{$\numquestions$} %command \numquestions fournie par le style `exam`
%\setNbPages{5} % optionnel, si non spécifié alors la valeur `\pageref{LastPage}` est utilisée


%% Calculatrice, indiquee si elle est autorisee ou interdite (obligatoire)
%\setCalculatriceAutorisee
\setCalculatriceInterdite

%% Si la calculatrice est autorisée, indiquée le type (une case sera cochée)
%\setTypeCalculatriceProgrammable
%\setTypeCalculatriceNonProgrammable
%\setTypeCalculatriceLogiciel

%% Indiquez le(s) modèle(s) de calculatrice
%\setCalculatriceModelesPermis{``TI-nspire CX CAS'' et ``TI-nspire CX II CAS''.}

%% Documentation, choisissez au moins une options et complétez si nécessaire.
%% La commande `\item` n'est pas optionnelle.
\setDocumentation{
  \item Aucune
  %\item feuille(s) de note(s), précisez le format et le nombre de pages
  %\item toutes documentation papier
  %\item documentation électronique, préciser :
  %\item documentation papier et électronique, précisez :
  %\item limitée, précisez :

  %% Exemples 
  %\item 1 feuille de note, au format lettre (2 pages en tout).
  %\item 3 tables de formules remise par l'enseignant(e) : tables de
  %  trigonométrie (feuille verte), formules de dérivation (feuille jaune),
  %  formules d'intégration (feuille bleue).
}


%% Annexes, choississez une ligne et complétez.
%% La commande `\item` n'est pas optionnelle.
\setAnnexes{
  \item Aucune
  %\item oui, préciser le titre et spécifiez les pages.
}

%% Inscrivez vos directives particulière ici. S'il n'y a pas, il faut quand même
%% appeler la commande avec un paramètre vide.
%\setDirectivesParticulieres{}

%% Décommentez si l'examen est en plusieurs parties. Indiquez les directives ici.
%\setExamenEnPlusieursParties{Deux parties : calculatrice est interdite pour la première partie, mais autorisée pour la deuxième. La durée maximale de la première partie : 45 minutes.}


%% Matériel informatique
%\setMaterielInformatiqueAutorise
\setMaterielInformatiqueInterdit

%% Si du matériel informatique est autorisée, il faut obligatoirement préciser de quoi il s'agit
%\setTypeMaterielInformatiqueOrdinateurPortable
%\setTypeMaterielInformatiqueTablette
%\setTypeMaterielInformatiqueCleUSB
%\setTypeMaterielInformatiqueAutre{Précisions ici}


%% Pour cacher les instruction s'adressant à la personne qui rédige l'examen
%\cacherInstructions


